\chapter{Folland. Measure}

\section{Exercises.1}

\begin{question}. A family of sets $\mathcal{R}\subset \mathcal{P}(X)$ is call \textbf{Ring}, if it is closed under unions and differences(i.e. if $E_1,\cdots,E_n\in \mathcal{R}$,then $\bigcup^n_{j=1}E_j\in \mathcal{R}$, if $E,F\in \mathcal{R}$,then $E/F\in \mathcal{R}$)\footnote{If a Ring countable for unions,then be called\textbf{$\sigma$-Ring}}
     \begin{enumerate}[itemindent=2em]
        \item Rings(resp. $\sigma$-rings) are closed under finite(resp. countable) intersections;
        \item If $\mathcal{R}$ is a Ring, then $\mathcal{R}$ is a \textbf{Algebra}\footnote{A family set $\mathcal{A}\in \mathcal{P}(X)$ has proporty as : $(1) X\in \mathcal{A},(2)E\in \mathcal{A},E^c\in \mathcal{A},(3) \forall E_i\in \mathcal{R},\bigcup^n_{i=0}E_i\in \mathcal{R}$ } iff $X\in \mathcal{R}$;
        \item If $\mathcal{R}$ is a $\sigma-ring$, then $\{E\subset X : E\in \mathcal{R}\ or\ E^c\in \mathcal{R}\}$ is a $\sigma-algebra$;
        \item If $\mathcal{R}$ is a $\sigma-ring$, then $\{E\subset X : E\cap F\in \mathcal{R}\ for\ all\ F\in \mathcal{R}\}$ is a $\sigma-algebra$;
     \end{enumerate}
\end{question}

\begin{mdframed}[backgroundcolor=gray!10,linewidth=0pt]
    \begin{pf}
        \begin{enumerate}
            \item $\forall E_1,E_2\in \mathcal{R}$,$E_1\cap E_2 = E_1/(E_1/E_2)\in \mathcal{R}$ by closed under differences ;
            \item if $X\in \mathcal{R}$, $\forall E\subset X,E\in \mathcal{R}$, $X/E=E^c\in \mathcal{R}$ by closed under differences;If $\mathcal{R}$ is a Algebra, the conclusion natually.
            \item Let's $\mathcal{S}:=\{E\subset X : E\in \mathcal{R}\ or\ E^c\in \mathcal{R}\}$,if $E\in \mathcal{S}$, $(E^c)^c\in \mathcal{S}\Rightarrow E^c\in \mathcal{S}$, and $X=E\cup E^c\in \mathcal{S}$ by closed under unions;
            \item Let's $\mathcal{S}:=\{E\subset X : E\cap F\in \mathcal{R}\ for\ all\ F\in \mathcal{R}\}$, 
            \begin{enumerate}
                \item Assume $E\in \mathcal{S}$, we prove $E^c\in \mathcal{S}$.
                
                if $E^c\in \mathcal{S}$,then $\forall F\in \mathcal{R}$, $E^c\cap F\in \mathcal{R}\Rightarrow F/(E\cap F)=\mathcal{R}$, then $E\cap F\in \mathcal{R}$ by closed under differences,so $E\in \mathcal{S}$;
                
                \item prove closed of unions; 
                
                $\forall E_1,E_2,\cdots,E_n\in \mathcal{S}$, If $\bigcup^n_{i=1} E_i\in \mathcal{S}$,then
                \begin{equation}
                    (\bigcup^n_{i=1} E_i)\cap F=\bigcup^n_{i=1} (E_i\cap F)\in \mathcal{R},\ \ \forall F\in \mathcal{R}
                \end{equation}

                So $(\bigcup^n_{i=1} E_i)\in \mathcal{S}$
            \end{enumerate}
        \end{enumerate}
    \end{pf}
\end{mdframed}

It is trivial to verify that the intersection of any family of $\sigma$-algebra on $X$ is again a $\sigma$-algebra. It follows that if $\mathcal{E}$ is any subset of $\mathcal{P}(X)$,
there is a \textbf{unique smallest $\sigma$-algebra $\mathcal{M}(\mathcal{E})$ contains $\mathcal{E}$}, $\mathcal{M}(\mathcal{E})$ is called the \textbf{$\sigma$-algebra generated by $\mathcal{E}$}.

If $X$ is any metric space, or more generally any topological space, the $\sigma$-algebra generated by the family of open sets (or, equivalently, by the family of closed sets in X) in $X$ is called the \textbf{\textsl{Borel $\sigma$-Algebra}} amd is denoted by $\mathcal{B}_{X}$.Its members are called \textbf{\textsl{Borel sets}}.

\begin{question}
    $\mathcal{B}_{\mathbb{R}}$ is generated by each of the following:
        \begin{enumerate}[itemindent=2em]
            \item the open intervals: $\mathcal{E}_1=\{(a,b):a<b\}$,
            \item the closed intervals: $\mathcal{E}_2=\{[a,b]:a<b\}$,
            \item the half-open intervals: $\mathcal{E}_3=\{(a,b]:a<b\}$ or $\mathcal{E}_4=\{[a,b):a<b\}$,
            \item the open rays: $\mathcal{E}_5=\{(a,\infty):a\in \mathbb{R}\}$ or $\mathcal{E}_6=\{(-\infty,a):a\in \mathbb{R}\}$,
            \item the closed rays: $\mathcal{E}_7=\{[a,\infty):a\in \mathbb{R}\}$ or $\mathcal{E}_8=\{(-\infty,a]:a\in \mathbb{R}\}$,
        \end{enumerate}
\end{question}

\begin{mdframed}[backgroundcolor=gray!10,linewidth=0pt]
    \begin{pf} Just prove 
        \begin{enumerate}
            \item prove $M(\mathcal{E}_j)\subset \mathcal{B}_{\mathbb{R}}$;
            
            $\mathcal{M}(\mathcal{E}_j)\subset \mathcal{B}_{\mathbb{R}}$ is natually for $j=1,2$ by definition. for $j=3,4$,the element of $\mathcal{E}_3$ and $\mathcal{E}_4$ are $G_\delta$ sets\footnote{A countable intersection of open sets is called a $G_\delta$ \textbf{sets};a countable unions of closed sets is called an $F_\delta$ \textbf{sets}.}
            , for example, 
            \begin{equation}
                (a,b]=\bigcap^\infty_1(a,b+\frac{1}{n})
            \end{equation}

            So all there are Borel sets, so $\mathcal{\mathcal{M}(\mathcal{E})}\subset \mathcal{B}_{\mathbb{R}}$.

            \item prove $\mathcal{B}_{\mathbb{R}}\subset M(\mathcal{E}_j)$;
            
            Every open set in $\mathbb{R}$ is a countable union of open intervals, so $\mathcal{B}_{\mathbb{R}}\subset \mathcal{M}(\mathcal{E}_1)$, for $j\geqslant 2$ can now be established by showing that all
            open intervals lie in $\mathcal{M}(\mathcal{E}_j)$, note that $\mathcal{M}(\mathcal{E_j})$ is $\sigma$-algebra, for example
            \begin{equation}
                (a,b)=\bigcup^\infty_1[a+\frac{1}{n},b-\frac{1}{n}]\in \mathcal{M}(\mathcal{E}_2)
            \end{equation}
            \begin{equation}
                (a,b)=\bigcap^\infty_1(a,b+\frac{1}{n}]\in \mathcal{M}(\mathcal{E}_3)
            \end{equation}
        \end{enumerate}
    \end{pf}
\end{mdframed}


\begin{question}
    Let $\mathcal{M}$ be an infinite $\sigma$-algebra.
    \begin{enumerate}[itemindent=2em]
        \item $\mathcal{M}$ contains an infinite sequence disjoint sets,
        \item $card(\mathcal{M})\geqslant c$.
    \end{enumerate}
\end{question}

\begin{mdframed}[backgroundcolor=gray!10,linewidth=0pt]
    \begin{pf} Assume $\mathcal{M}\subset \mathcal{P}(X)$ is a $\sigma$-algebra
        \begin{enumerate}[itemindent=2em]
            \item 
        \end{enumerate}
    \end{pf}
\end{mdframed}

\begin{question}
    An algebra $\mathcal{A}$ is a $\sigma$-algebra iff $\mathcal{A}$ is closed under countable increasing unions(i.e. if $\{E_j\}^\infty_1\subset \mathcal{A}$ and $E_1\subset E_2\subset \cdots, then \bigcup^\infty_1E_j\in\mathcal{A}$)
\end{question}

\begin{mdframed}[backgroundcolor=gray!10,linewidth=0pt]
    \begin{pf} 
    \end{pf}
\end{mdframed}

\begin{question}
    If $\mathcal{M}$ is the $\sigma$-algebra generated by $\mathcal{E}$, then $\mathcal{M}$ is the union of the $\sigma$-algebras generated by $\mathcal{F}$ as $\mathcal{F}$ ranges over all countable subsets of $\mathcal{E}$.(Hint: Show that the latter object is a $\sigma$-algebra).
\end{question}

\begin{mdframed}[backgroundcolor=gray!10,linewidth=0pt]
    \begin{pf} 
    \end{pf}
\end{mdframed}